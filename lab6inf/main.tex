
\documentclass[a4paper,12pt]{article}
\usepackage[utf8]{inputenc}    % Поддержка UTF-8
\usepackage[T2A]{fontenc}      % Кириллическая кодировка
\usepackage[russian]{babel}    % Русский язык
\usepackage{amsmath, amssymb}
\usepackage[dvipsnames, svgnames, x11names]{xcolor}
\usepackage{tikz}
\usepackage{geometry}
\usepackage{amsmath}
\usepackage{mathtools}
\usepackage{fancyhdr}          % Для кастомных заголовков
\usepackage{graphicx}
\usepackage{float}
\usepackage{wrapfig}
\geometry{top=2cm, bottom=2cm, left=1cm, right=2cm}
\setlength{\parindent}{0pt}

\thispagestyle{empty}
\begin{document}
\begin{minipage}{0.25\textwidth}
\vspace{-15cm}
\begin{tikzpicture}[scale=1.3][t]
  \coordinate (D) at (0, 0)
    \coordinate (C) at (4, 0)
    \coordinate (B) at (2, 3.464)
    \coordinate (A) at (2, 0);
    \fill[yellow!100] (1.8, 0) -- (2.8, 0) arc (0:180:0.8) -- cycle;
    \fill[orange!100] (2, 0) -- (2.8,0) arc (0:90:0.8) -- cycle;
    \draw[line width=1.2mm, NavyBlue] (B) -- (A);
    \draw[line width=1.2mm, Bittersweet] (B) -- (C);
    \draw[line width=1.2mm, black] (A) -- (C);
    \draw[dashed, line width=1.2mm, black] (D) -- (A);
    \draw[dashed,line width=1.2mm, Bittersweet] (D) -- (B);

    \node[below left] at (D) {\small{D}};
    \node[above] at (B) {\small{B}};
    \node[below right] at (C) {\small{C}};
    \node[below] at (A) {\small{A}};
\end{tikzpicture}
\end{minipage}
\hspace{0.1cm}
\hspace{2cm}
\begin{minipage}{0.65\textwidth}
\textbf{74}
\hspace{1 cm}
\textsc{\textbf{КНИГА I ПРЕДЛ. XLVIII. ТЕОРЕМА}}\\
\begin{wrapfigure}[]{l}{2cm}
    \vspace{-0.5cm}
    \includegraphics[width=2cm]{ff.png} % Путь к файлу изображения
\end{wrapfigure}
\textit{сли в треугольнике квадрат одной стороны} 
\begin{tikzpicture}
  % Рисуем отрезок
  \draw[line width = 0.1cm, orange] (0,0) -- ++ (1.5,0);  % отрезок с сдвигом
  \node[above]{B};
  \node[above] at (1.5, 0){C};
\end{tikzpicture}
\textit{равен сумме квадратов двух других сторон}
\begin{tikzpicture}
    \draw[line width = 0.1cm, blue] (0,0) -- ++ (2,0);
    \node[above]{A};
    \node[above] at (2, 0){B};
\end{tikzpicture}
\textit{и}
\begin{tikzpicture}
    \draw[line width = 0.1cm, black] (0,0) -- ++ (2,0);
    \node[above]{A};
    \node[above] at (2, 0){C};
\end{tikzpicture}
\textit{, то угол}
\begin{tikzpicture}
    \fill[orange!100](0,0) -- (1,0) arc(0:90:1) -- cycle;
    \node at (-0.07, 1.1) {\tiny{B}};
    \node at (-0.2,-0.1) {\tiny{A}};
    \node at (1.2, -0.05)  {\tiny{C}};
\end{tikzpicture}
\textit{, заключенный между этими двумя сторонами и прямой.}

\vspace{0.3cm}
\begin{center}


Проведем 
 \begin{tikzpicture}
    \draw[dashed, line width=0.1cm, black] (0,-0.5) -- ++ (1.5, 0);
    \node at (0,-0.15){A};
    \node at (1.5, -0.15) {D};
\end{tikzpicture}
$\perp$
\begin{tikzpicture}
    \draw[line width=0.1cm, blue] (0,-0.5) -- ++ (1.5, 0);
    \node at (0,-0.15){A};
    \node at (1.5, -0.15) {B};
\end{tikzpicture} \\
и \textbf{=}
\begin{tikzpicture}
    \draw[line width=0.1cm, blue] (0,-0.5) -- ++ (1.5, 0);
    \node at (0,-0.15){A};
    \node at (1.5, -0.15) {C};
\end{tikzpicture} 
(пр.)\\
также проведем 
\begin{tikzpicture}
    \draw[dashed, line width=0.1cm, orange] (0,-0.5) -- ++ (1.5, 0);
    \node at (0,-0.15){B};
    \node at (1.5, -0.15) {D};
\end{tikzpicture} \\
\vspace{0.7cm}
Поскольку 
\begin{tikzpicture}
    \draw[dashed,line width=0.1cm, blue] (0,-0.5) -- ++ (1.5, 0);
    \node at (0,-0.15){A};
    \node at (1.5, -0.15) {D};
\end{tikzpicture} 
=
\begin{tikzpicture}
    \draw[line width=0.1cm, blue] (0,-0.5) -- ++ (1.5, 0);
    \node at (0,-0.15){A};
    \node at (1.5, -0.15) {C};
\end{tikzpicture} 
(постр.)\\
\begin{tikzpicture}
    \draw[dashed, line width=0.1cm, blue] (0,-0.5) -- ++ (1.5, 0);
    \node at (0,-0.15){A};
    \node at (1.5, -0.15) {D};
     \node at (1.7, 0.1) {\tiny{2}};
\end{tikzpicture} 
\large{=}
\begin{tikzpicture}
    \draw[line width=0.1cm, blue] (0,-0.5) -- ++ (1.5, 0);
    \node at (0,-0.15){A};
    \node at (1.5, -0.15) {C};
    \node at (1.7, 0.1) {\tiny{2}};
\end{tikzpicture} 
;\\
\vspace{0.7cm}
\therefore
\begin{tikzpicture}
    \draw[dashed, line width=0.1cm, black] (0,-0.5) -- ++ (1.5, 0);
    \node at (0,-0.15){A};
    \node at (1.5, -0.15) {D};
    \node at (1.7, 0.1) {\tiny{2}};
\end{tikzpicture} 
\large{+}
\begin{tikzpicture}
    \draw[line width=0.1cm, blue] (0,-0.5) -- ++ (1.5, 0);
    \node at (0,-0.15){A};
    \node at (1.5, -0.15) {B};
    \node at (1.7, 0.1) {\tiny{2}};
\end{tikzpicture} 
\large{=}
\begin{tikzpicture}
    \draw[line width=0.1cm, black] (0,-0.5) -- ++ (1.5, 0);
    \node at (0,-0.15){A};
    \node at (1.5, -0.15) {C};
    \node at (1.7, 0.1) {\tiny{2}};
\end{tikzpicture} 
\large{+}
\begin{tikzpicture}
    \draw[line width=0.1cm, blue] (0,-0.5) -- ++ (1.5, 0);
    \node at (0,-0.15){A};
    \node at (1.5, -0.15) {B};
    \node at (1.7, 0.1) {\tiny{2}};
\end{tikzpicture} \\
\vspace{0.7cm}
но
\begin{tikzpicture}
    \draw[dashed, line width=0.1cm, black] (0,-0.5) -- ++ (1.5, 0);
    \node at (0,-0.15){A};
    \node at (1.5, -0.15) {D};
    \node at (1.7, 0.1) {\tiny{2}};
\end{tikzpicture} 
\large{+}
\begin{tikzpicture}
    \draw[line width=0.1cm, blue] (0,-0.5) -- ++ (1.5, 0);
    \node at (0,-0.15){A};
    \node at (1.5, -0.15) {B};
    \node at (1.7, 0.1) {\tiny{2}};
\end{tikzpicture} 
\large{=}
\begin{tikzpicture}
    \draw[dashed, line width=0.1cm, orange] (0,-0.5) -- ++ (1.5, 0);
    \node at (0,-0.15){B};
    \node at (1.5, -0.15) {D};
    \node at (1.7, 0.1) {\tiny{2}};
\end{tikzpicture} 
(пр.I.47), \\
и 
\begin{tikzpicture}
    \draw[line width=0.1cm, black] (0,-0.5) -- ++ (1.5, 0);
    \node at (0,-0.15){A};
    \node at (1.5, -0.15) {C};
    \node at (1.7, 0.1) {\tiny{2}};
\end{tikzpicture} 
\large{+}
\begin{tikzpicture}
    \draw[line width=0.1cm, blue] (0,-0.5) -- ++ (1.5, 0);
    \node at (0,-0.15){A};
    \node at (1.5, -0.15) {B};
    \node at (1.7, 0.1) {\tiny{2}};
\end{tikzpicture} 
\large{=}
\begin{tikzpicture}
    \draw[line width=0.1cm, orange] (0,-0.5) -- ++ (1.5, 0);
    \node at (0,-0.15){B};
    \node at (1.5, -0.15) {C};
    \node at (1.7, 0.1) {\tiny{2}};
\end{tikzpicture} 
(гип.)\\
\vspace{0.7cm}
\therefore
\begin{tikzpicture}
    \draw[dashed, line width=0.1cm, orange] (0,-0.5) -- ++ (1.5, 0);
    \node at (0,-0.15){B};
    \node at (1.5, -0.15) {D};
    \node at (1.7, 0.1) {\tiny{2}};
\end{tikzpicture} 
\large{=}
\begin{tikzpicture}
    \draw[line width=0.1cm, orange] (0,-0.5) -- ++ (1.5, 0);
    \node at (0,-0.15){B};
    \node at (1.5, -0.15) {C};
    \node at (1.7, 0.1) {\tiny{2}};
\end{tikzpicture}, \\
\therefore
\begin{tikzpicture}
    \draw[dashed, line width=0.1cm, orange] (0,-0.5) -- ++ (1.5, 0);
    \node at (0,-0.15){B};
    \node at (1.5, -0.15) {D};
\end{tikzpicture} 
\large{=}
\begin{tikzpicture}
\draw[line width=0.1cm, orange] (0,-0.5) -- ++ (1.5, 0);
    \node at (0,-0.15){B};
    \node at (1.5, -0.15) {C};
\end{tikzpicture}; \\
\vspace{0.7cm}
и 
\therefore
\begin{tikzpicture}
        \fill[yellow!100](1,0) -- (0,0) arc(180:90:1) -- cycle;
        \node at (1.1, 1.1) {\tiny{B}};
        \node at (-0.2,-0.1) {\tiny{D}};
        \node at (1.2, -0.05)  {\tiny{A}};
\end{tikzpicture}
\large{=}
\begin{tikzpicture}
        \fill[orange!100](0,0) -- (1,0) arc(0:90:1) -- cycle;
            \node at (-0.07, 1.1) {\tiny{B}};
    \node at (-0.2,-0.1) {\tiny{A}};
    \node at (1.2, -0.05)  {\tiny{C}};
\end{tikzpicture}
(пр.I.8),\\
следовательно 
\begin{tikzpicture}
        \fill[orange!100](0,0) -- (1,0) arc(0:90:1) -- cycle;
            \node at (-0.07, 1.1) {\tiny{B}};
    \node at (-0.2,-0.1) {\tiny{A}};
    \node at (1.2, -0.05)  {\tiny{C}};
\end{tikzpicture}
прямой угол
\end{center}
\end{minipage}
\vspace{1cm}
\begin{flushright}
\textbf{ч.т.д.}
\end{flushright}

\end{document}